
\documentclass{beamer}

\usepackage{luatexja}
\usepackage[ipaex]{luatexja-preset}
\usepackage{mathtools,amsmath, amssymb, amsfonts, amsthm,bm,float,graphicx,xcolor}

\usetheme{Copenhagen}
\usefonttheme{professionalfonts}
\renewcommand{\kanjifamilydefault}{\gtdefault}

%%% amsthm関係 %%%%%%%%%%%%%%%%%%%%%%%%%%%%%
% \begin{thm}[name] \end{thm} のように使えます

\setbeamertemplate{theorems}[numbered]
\theoremstyle{definition}
\newtheorem{thm}{Theorem}[section]
\newtheorem{lem}[thm]{Lemma}
\newtheorem{prop}[thm]{Proposition}
\newtheorem{cor}[thm]{Corollary}
\newtheorem{conj}[thm]{Conjecture}
\newtheorem{ass}[thm]{Assumption}
\newtheorem{dfn}[thm]{Definition}

\theoremstyle{remark}
\newtheorem{rem}[thm]{Remark}

%%% タイトル %%%%%%%%%%%%%%%%%%%%%%%%%%%%%%%%%

\title{スライド名をここに入力}
\subtitle{サブタイトルをここに入力}
\author{あなたの名前をここに入力}
\date{\today}


%%% 本文 %%%%%%%%%%%%%%%%%%%%%%%%%%%%%%%%%%%%
\begin{document}

\frame{\titlepage} % タイトルページ
\frame{\tableofcontents} % 目次ページ

\section{セクション名をここに入力}
\subsection{サブセクション名をここに入力}

\begin{frame}{ページタイトルをここに入力}
    Hello World! 正しく表示されていますか?この部分を適当に書き換えた後に緑の再生ボタンを押して、変更が反映されるか確認しましょう。

    \begin{block}{ブロックタイトルをここに入力}
        ブロック内の記述
    \end{block}

\end{frame}



\end{document}
%%%%%%%%%%%%%%%%%%%%%%%%%%%%%%%%%%%%%%%%%%%%